\chapter{Graded Assignments}

\newpage
\section{Assignment 1}

\textbf{Instructions: }
\begin{enumerate}
    \item The assignment should be hand in on Brigthspace in only one file with the name ``YourSurname\_Assignment1.mlx''.
    \item Every function should have an initial multiline comment where you explain what the function does, the inputs and the exact outputs. 
    \item Every key step of the algorithm should be commented. If you also modified something from the standard algorithms, modifications should be explained. 
    \item You are allowed to ask any question that you have regarding the assignment to the instructor. 
    \item You are allowed to work with other students in groups of at most 3 people. In case you do, you must in the end hand in your own code with the solutions and add at the beginning of your code the names of the students you worked with.
\end{enumerate}


\begin{exercise}
    Create a function that helps you estimate the order of convergence of a method visually. It must take as inputs
    \begin{enumerate}
        \item An array representing the $M$ first terms $\{p_n\}_{n=1}^{M}$ of a convergent series.
        \item A minimum value $\alpha_{min}$.
        \item A maximmum value $\alpha_{max}$.
        \item An integer $N$ representing how many different orders $\alpha$ you want to test.
    \end{enumerate} 
    And it must make a plot with the $N$ different sequences $\{q^\alpha_n\}_{n=1}^{M-1}$ defined by 
    \begin{equation*}
        q^\alpha_n := \frac{|p_{n+1}-p|}{|p_n-p|^\alpha},
    \end{equation*}
    where $\alpha$ are $N$ equispaced values between $\alpha_{min}$ and $\alpha_{max}$. The plot should have a legend that identifies each line with its value of $\alpha$.
\end{exercise}

\begin{exercise}
    Modify Bisection, Newtons, Secant, False position and Modified Newtons Algorithm to return not only the final result for the root of a function, but also the sequence generated by the method. 
\end{exercise}

\begin{exercise}
    Use the methods in the last exercise to find the Root that is in the interval $[0,2]$ with an accuracy of $10^{-10}$ for the functions
    \begin{enumerate}
        \item $g_1(x) = \cos(x)$
        \item $g_2(x) = \cos^2(x)$
        \item $g_3(x) = \cos^3(x)$
    \end{enumerate}
\end{exercise}

\begin{exercise}
    For each of the functions and methods in the previous Exercise 5.1.3 use the function of Exercise 5.1.1 to estimate the order of convergence visually. If you cannot do it explain your conclusions. 
\end{exercise}

\begin{exercise}
    For each of the functions and methods in the previous Exercise 5.1.3 plot the sequences of estimations given by each method. Order the methods from fastest to slowest for each case. If something failed, explain why. Be careful to choose a good scale to compare them.
\end{exercise}




\newpage
\section{Assignment 2} 

\textbf{Instructions: }
\begin{enumerate}
    \item The assignment should be hand in on Brigthspace in only one file with the name ``YourSurname\_Assignment2.mlx''.
    \item Every function should have an initial multiline comment where you explain what the function does, the inputs and the exact outputs. 
    \item Every key step of the algorithm should be commented. If you also modified something from the standard algorithms, modifications should be explained. 
    \item You are allowed to ask any question that you have regarding the assignment to the instructor. 
    \item You are allowed to work with other students in groups of at most 3 people. In case you do, you must in the end hand in your own code with the solutions and add at the beginning of your code the names of the students you worked with.
\end{enumerate}


\begin{exercise}
    Modify the Divided differences Algorithm for polynomial interpolation for the case that you want to find a polynomial that satisfies
    \begin{align*}
        P^{(k_{i,j})}(x_i) = f^{(k_{i,j})}(x_i) \text{ for } j=0,\dots,k_i, i=1,\dots,n.  
    \end{align*}
    That is, we want the polynomial to have the same values and derivatives as the function at the points. But the amount of derivatives change for every interpolating node. 
\end{exercise}

\begin{exercise} \textcolor{red}{In progress}
    
\end{exercise}
